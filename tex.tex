\section*{第1問 式(5.18)から式(5.19)の導出}

ベクトルポテンシャル $\mathbf{A}$ はローレンツ・ゲージのもとで次式に従う:
\begin{equation}
(\nabla^2 + k^2)\mathbf{A} = -\mu \mathbf{i}
\label{eq:518}
\end{equation}

また,ローレンツ条件より
\begin{equation}
\nabla \cdot \mathbf{A} + j\omega\mu\varepsilon V = 0
\quad\Rightarrow\quad
\nabla \cdot \mathbf{A} = -j\omega\mu\varepsilon V
\label{eq:Lorenz}
\end{equation}

時間調和場(時間依存性 $e^{j\omega t}$)を仮定すると,連続の式は
\begin{equation}
\nabla \cdot \mathbf{i} + j\omega \rho = 0
\quad\Rightarrow\quad
\nabla \cdot \mathbf{i} = -j\omega \rho
\label{eq:continuity}
\end{equation}

式\eqref{eq:518} の両辺に発散をとる:
\begin{align}
\nabla \cdot \bigl[(\nabla^2 + k^2)\mathbf{A}\bigr]
&= -\mu \,\nabla \cdot \mathbf{i}
\end{align}

ベクトル演算の交換可能性より
\begin{equation}
\nabla \cdot (\nabla^2 \mathbf{A}) = \nabla^2 (\nabla \cdot \mathbf{A})
\end{equation}
であるから,
\begin{align}
\nabla^2 (\nabla \cdot \mathbf{A}) + k^2 (\nabla \cdot \mathbf{A})
= -\mu \,\nabla \cdot \mathbf{i}
\label{eq:div_step}
\end{align}

ここでローレンツ条件 \eqref{eq:Lorenz} を用いると
\begin{align}
\nabla^2 (\nabla \cdot \mathbf{A}) + k^2 (\nabla \cdot \mathbf{A})
&= \nabla^2(-j\omega\mu\varepsilon V) + k^2(-j\omega\mu\varepsilon V) \notag\\
&= -j\omega\mu\varepsilon(\nabla^2 V + k^2 V)
\end{align}
したがって \eqref{eq:div_step} は
\begin{equation}
- j\omega\mu\varepsilon(\nabla^2 V + k^2 V)
= -\mu \,\nabla \cdot \mathbf{i}
\end{equation}
両辺を $-\mu$ で割ると
\begin{equation}
j\omega\varepsilon(\nabla^2 V + k^2 V) = \nabla \cdot \mathbf{i}
\end{equation}

さらに連続の式 \eqref{eq:continuity} を代入すると
\begin{equation}
j\omega\varepsilon(\nabla^2 V + k^2 V) = -j\omega \rho
\end{equation}
両辺を $j\omega$ で割り整理して
\begin{equation}
\varepsilon(\nabla^2 V + k^2 V) = -\rho
\end{equation}
よって
\begin{equation}
(\nabla^2 + k^2)V = -\frac{\rho}{\varepsilon}
\label{eq:519}
\end{equation}
が得られ,式(5.19)が式(5.18)から導かれることが示された。

\newpage

\section*{第2問 (演習問題5.2)式(5.24)の $V$ が式(5.19)を満たすことの証明}

演習5.2では,点電荷からのスカラポテンシャルとして
\begin{equation}
V(r) = c_1 \frac{e^{-jkr}}{r}
\label{eq:524}
\end{equation}
という形を仮定し,これがヘルムホルツ方程式
\begin{equation}
(\nabla^2 + k^2)V = -\frac{\rho}{\varepsilon}
\label{eq:Helm}
\end{equation}
を満たすことを確かめる(特に電荷分布のない領域での振る舞いを確認する)。

点電荷の位置から離れた領域では $\rho = 0$ であるから,
式\eqref{eq:Helm} は
\begin{equation}
(\nabla^2 + k^2)V = 0
\label{eq:Helm_free}
\end{equation}
となる。この式を,\eqref{eq:524} に対して満たすことを示す。

球対称なスカラ関数 $V(r)$ に対して,ラプラシアンは
\begin{equation}
\nabla^2 V
= \frac{1}{r^2}\frac{d}{dr}\left( r^2 \frac{dV}{dr} \right)
\label{eq:spherical_lap}
\end{equation}
で与えられる。

$V(r) = c_1 e^{-jkr}/r$ を式\eqref{eq:spherical_lap} に代入して計算する。

まず一次微分:
\begin{align}
\frac{dV}{dr}
&= c_1 \frac{d}{dr}\left(\frac{e^{-jkr}}{r}\right) \notag\\
&= c_1\left(
      \frac{-jk e^{-jkr}}{r}
      - \frac{e^{-jkr}}{r^2}
    \right)
\end{align}

次に $r^2 \dfrac{dV}{dr}$ を求める:
\begin{align}
r^2 \frac{dV}{dr}
&= c_1 e^{-jkr}(-jk r - 1)
\end{align}

これを $r$ で微分すると
\begin{align}
\frac{d}{dr}\left(r^2 \frac{dV}{dr}\right)
&= c_1 \frac{d}{dr}\left[ e^{-jkr}(-jk r - 1) \right] \notag\\
&= c_1\left[
   (-jk e^{-jkr})(-jk r - 1)
   + e^{-jkr}(-jk)
\right]
\end{align}

括弧内を整理する:
\begin{align}
(-jk)(-jk r - 1) - jk
&= (-jk)(-jk r) + (-jk)(-1) - jk \notag\\
&= (-1)k^2 r + jk - jk \notag\\
&= -k^2 r
\end{align}

よって
\begin{equation}
\frac{d}{dr}\left(r^2 \frac{dV}{dr}\right)
= -c_1 k^2 r e^{-jkr}
\end{equation}

これを式\eqref{eq:spherical_lap} に代入すると
\begin{align}
\nabla^2 V
&= \frac{1}{r^2}\left( -c_1 k^2 r e^{-jkr} \right) \notag\\
&= -k^2 c_1 \frac{e^{-jkr}}{r} \notag\\
&= -k^2 V
\end{align}

したがって
\begin{equation}
(\nabla^2 + k^2)V
= (-k^2 V) + k^2 V
= 0
\end{equation}
となり,$\rho = 0$ の領域(点電荷の外側)において
\eqref{eq:524} はヘルムホルツ方程式 \eqref{eq:Helm_free} を満たすことが示された。

さらに,静的極限 $k \to 0$ では
\begin{equation}
V(r) \xrightarrow{k \to 0} \frac{c_1}{r}
\end{equation}
一方,静電界における点電荷 $q$ のポテンシャルは
\begin{equation}
V(r) = \frac{q}{4\pi\varepsilon} \frac{1}{r}
\end{equation}
であるから,両者を一致させるため
\begin{equation}
c_1 = \frac{q}{4\pi\varepsilon}
\end{equation}
となる。従って
\begin{equation}
V(r) = \frac{q}{4\pi\varepsilon}\frac{e^{-jkr}}{r}
\end{equation}
が得られ,これが式(5.24)に対応し,かつヘルムホルツ方程式(5.19)を正しく満たしていることが確認できる。
